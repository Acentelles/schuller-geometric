
Lie groups and Lie algebras are used in physics mostly in terms of what are called representations. Very often they are even defined in terms of their concrete representations. We took a more abstract approach by defining a Lie group as a smooth manifold with a compatible group structure, and its associated Lie algebra as the space of left-invariant vector fields, which we then showed to be isomorphic to the tangent space at the identity.

\subsection{Representations of Lie algebras}

\bd
Let $L$ be a Lie algebra. A \emph{representation}\index{representation} of $L$ is a Lie algebra homomorphism
\bse
\rho\cl L \xrightarrow{\sim} \End(V),
\ese
where $V$ is some finite-dimensional vector space over the same field as $L$.
\ed
Recall that a linear map $\rho\cl L \xrightarrow{\sim} \End(V)$ is a Lie algebra homomorphism if
\bse
\forall \, x,y\in L : \ \rho([x,y]) = [\rho(x),\rho(y)]:=\rho(x)\circ\rho(y)-\rho(y)\circ\rho(x),
\ese
where the right hand side is the natural Lie bracket on $\End(V)$.

\bd
Let $\rho\cl L \xrightarrow{\sim}\End(V)$ be a representation of $L$.
\ben[label=\roman*)]
\item The vector space $V$ is called the \emph{representation space} of $\rho$.
\item The \emph{dimension}\index{dimension!representation} of the representation $\rho$ is $\dim V$.
\een
\ed

\be
Consider the Lie algebra $\sl(2,\C)$. We constructed a basis $\{X_1,X_2,X_3\}$ satisfying the relations
\bi{rCl}
[X_1,X_2] & = & 2X_2,\\
{[X_1,X_3]} & = & -2X_3,\\
{[X_2,X_3]} & = & X_1.
\ei
Let $\rho\cl\sl(2,\C)\xrightarrow{\sim}\End(\C^2)$ be the linear map defined by
\bse
\rho(X_1) := \biggl(\begin{matrix}1& 0\\ 0 & -1\end{matrix}\biggr), \qquad \rho(X_2) := \biggl(\begin{matrix}0& 1\\ 0 & 0\end{matrix}\biggr), \qquad \rho(X_3) := \biggl(\begin{matrix}0& 0\\ 1 & 0\end{matrix}\biggr)
\ese
(recall that a linear map is completely determined by its action on a basis, by linear continuation). To check that $\rho$ is a representation of $\sl(2,\C)$, we calculate
\bi{rCl}
[\rho(X_1),\rho(X_2)] & = & \biggl(\begin{matrix}1& 0\\ 0 & -1\end{matrix}\biggr)\biggl(\begin{matrix}0& 1\\ 0 & 0\end{matrix}\biggr)-\biggl(\begin{matrix}0& 1\\ 0 & 0\end{matrix}\biggr)\biggl(\begin{matrix}1& 0\\ 0 & -1\end{matrix}\biggr)\\
%& = & \biggl(\begin{matrix}0& 1\\ 0 & 0\end{matrix}\biggr)-\biggl(\begin{matrix}0& -1\\ 0 & 0\end{matrix}\biggr)\\
& = & \biggl(\begin{matrix}0& 2\\ 0 & 0\end{matrix}\biggr)\\
& = &\rho(2X_2)\\
& = &\rho([X_1,X_2]).
\ei
Similarly, we find
\bi{rCl}
[\rho(X_1),\rho(X_3)] & = & \rho([X_1,X_3]),\\
{[\rho(X_2),\rho(X_3)]} & = & \rho([X_2,X_3]).
\ei
By linear continuation, $\rho([x,y]) = [\rho(x),\rho(y)]$ for any $x,y\in \sl(2,\C)$ and hence, $\rho$ is a $2$-dimensional representation of $\sl(2,\C)$ with representation space $\C^2$. Note that we have
\bi{rCl}
\im_\rho(\sl(2,\C)) & = & \biggl\{ \biggl(\begin{matrix}a& b\\ c & d\end{matrix}\biggr) \in \End(\C^2) \ \Big| \ a+d = 0 \biggr\}\\[3pt]
& = & \{\phi\in\End(\C^2)\mid \tr \phi = 0\}.
\ei
This is how $\sl(2,\C)$ is often defined in physics courses, i.e.\ as the algebra of $2\times 2$ complex traceless matrices.
\ee

Two representations of a Lie algebra can be related in the following sense.

\bd
Let $L$ be a Lie algebra and let 
\bse
\rho_1\cl L \xrightarrow{\sim} \End(V_1), \qquad 
\rho_2\cl L \xrightarrow{\sim} \End(V_2)
\ese
be representations of $L$. A linear map $f\cl V_1\xrightarrow{\sim}V_2$ is a \emph{homomorphism of representations} if
\bse
\forall \, x \in L : \ f\circ \rho_1(x) = \rho_2(x)\circ f.
\ese
Equivalently, if the following diagram commutes for all $x\in L$.
\bse
\begin{tikzcd}
V_1 \ar[rr,"f"] \ar[dd,"\rho_1(x)"']&& V_2\ar[dd,"\rho_2(x)"]\\
&&\\
V_1\ar[rr,"f"] && V_2
\end{tikzcd}
\ese
\ed
If in addition $f\cl V_1\xrightarrow{\sim}V_2$ is a linear isomorphism, then $f^{-1}\cl V_2\xrightarrow{\sim}V_1$ is automatically a homomorphism of representations, since
\bi{rCl}
f\circ \rho_1(x) = \rho_2(x)\circ f &\ \Leftrightarrow\ & f^{-1}\circ (f\circ \rho_1(x)) \circ f^{-1} = f^{-1}\circ(\rho_2(x)\circ f)\circ f^{-1} \\
& \Leftrightarrow & \rho_1(x) \circ f^{-1} = f^{-1}\circ\rho_2(x).
\ei
\bd
An \emph{isomorphism of representations}\index{isomorphism!of representations} of Lie algebras is a bijective homomorphism of representations.
\ed
Isomorphic representations necessarily have the same dimension.
\be
Consider $\so(3,\R)$, the Lie algebra of the rotation group $\SO(3,\R)$. It is a $3$-dimensional Lie algebra over $\R$. It has a basis $\{J_1,J_2,J_3\}$ satisfying
\bse
[J_i,J_j] = C^{k}_{\phantom{k}ij} J_k,
\ese
where the structure constants $C^{k}_{\phantom{k}ij}$ are defined by first ``pulling the index $k$ down'' using the Killing form $\kappa_{ab}=C^{m}_{\phantom{m}an} C^{n}_{\phantom{n}bm}$ to obtain $C_{kij}:=\kappa_{km} C^{m}_{\phantom{m}ij}$, and then setting
\bse
C_{kij}:= \varepsilon_{ijk} := \begin{cases}\ 1 & \text{ if $(i\, j\, k)$ is an even permutation of $(1\, 2\, 3)$} \\
-1 & \text{ if $(i\, j\, k)$ is an odd permutation of $(1\, 2\, 3)$}\\
\ 0 & \text{ otherwise}.\end{cases}
\ese
By evaluating these, we find
\bi{rCl}
[J_1,J_2] & = & J_3,\\
{[J_2,J_3]} & = & J_1,\\
{[J_3,J_1]} & = & J_2.
\ei
Define a linear map $\rho_{\mathrm{vec}}\cl\so(3,\R)\xrightarrow{\sim}\End(\R^3)$ by
\bse
\rho_{\mathrm{vec}}(J_1) := \begin{pmatrix}0 & 0 & 0\\ 0 & 0 & -1\\ 0 & 1 & 0\end{pmatrix}, \qquad \rho_{\mathrm{vec}}(J_2) := \begin{pmatrix}0 & 0 & 1\\ 0 & 0 & 0\\ -1 & 0 & 0\end{pmatrix}, \qquad \rho_{\mathrm{vec}}(J_3) :=\begin{pmatrix}0 & -1 & 0\\ 1 & 0 & 0\\ 0 & 0 & 0\end{pmatrix}.
\ese
You can easily check that this is a representation of $\so(3,\R)$. However, as you may be aware from quantum mechanics, there is another representation of $\so(3,\R)$, namely
\bse
\rho_{\mathrm{spin}}\cl\so(3,\R)\xrightarrow{\sim}\End(\C^2),
\ese
with $\C^2$ understood as a $4$-dimensional $\R$-vector space, defined by
\bse
\rho_{\mathrm{spin}}(J_1) := -\frac{\mathrm{i}}{2}\, \sigma_1, \qquad \rho_{\mathrm{spin}}(J_2) := -\frac{\mathrm{i}}{2}\, \sigma_2, \qquad \rho_{\mathrm{spin}}(J_3) := -\frac{\mathrm{i}}{2}\, \sigma_3,
\ese
where $\sigma_1,\sigma_2,\sigma_3$ are the Pauli matrices
\bse
\sigma_1 = \biggl(\begin{matrix}0& 1\\ 1 & 0\end{matrix}\biggr), \qquad \sigma_2 = \biggl(\begin{matrix}0& -\mathrm{i}\\ \mathrm{i} & 0\end{matrix}\biggr), \qquad
\sigma_3 = \biggl(\begin{matrix}1& 0\\ 0 & -1\end{matrix}\biggr).
\ese
You can again check that this is a representation of $\so(3,\R)$. Since
\bse
\dim \R^3 = 3 \neq 4 = \dim \C^2,
\ese
the representations $\rho_{\mathrm{vec}}$ and $\rho_{\mathrm{spin}}$ are not isomorphic. 
\ee
Any (non-abelian) Lie algebra always has at least two special representations.

\bd
Let $L$ be a Lie algebra.  A \emph{trivial representation} of $L$ is defined by
\bi{rrCl}
\rho_{\mathrm{trv}} \cl & L & \xrightarrow{\sim} & \End(V)\\
& x & \mapsto & \rho_{\mathrm{trv}}(x) := 0,
\ei
where $0$ denotes the trivial endomorphism on $V$.
\ed
\bd
The \emph{adjoint representation}\index{adjoint representation} of $L$ is
\bi{rrCl}
\rho_{\mathrm{adj}} \cl & L & \xrightarrow{\sim} & \End(L)\\
& x & \mapsto & \rho_{\mathrm{adj}}(x) := \ad(x).
\ei
\ed

These are indeed representations since we have already shown that $\ad$ is a Lie algebra homomorphism, while for the trivial representations we have
\bse
\forall\, x,y \in L : \ \rho_\mathrm{trv}([x,y]) = 0 = [\rho_\mathrm{trv}(x),\rho_\mathrm{trv}(y)].
\ese

\bd
A representation $\rho\cl L \xrightarrow{\sim} \End(V)$ is called \emph{faithful} if $\rho$ is injective, i.e.\
\bse
\dim(\im_\rho(L)) = \dim L.
\ese
\ed

\be
All representations considered so far are faithful, except for the trivial representations whenever the Lie algebra $L$ is not itself trivial. Consider, for instance, the adjoint representation. We have
\bi{rCl}
\ad(x) = \ad(y) & \Leftrightarrow & \forall \, z \in L :  \ad(x)z = \ad(y)z\\
 & \Leftrightarrow & \forall \, z \in L :  [x,z] = [y,z]\\
 & \Leftrightarrow & \forall \, z \in L :  [x-y,z] = 0.
\ei
If $L$ is trivial, then any representation is faithful. Otherwise, there is some non-zero $z\in L$, hence we must have $x-y=0$, so $x=y$, and thus $\ad$ is injective.
\ee

\bd
Given two representations $\rho_1\cl L \xrightarrow{\sim} \End(V_1)$,  $\rho_2\cl L \xrightarrow{\sim} \End(V_2)$, we can construct new representations called
\ben[label=\roman*)]
\item the \emph{direct sum representation}
\bi{rrCl}
\rho_1\oplus \rho_2 \cl & L &\xrightarrow{\sim} &\End(V_1\oplus V_2)\\
& x & \mapsto & (\rho_1\oplus \rho_2) (x)  := \rho_1(x)\oplus \rho_2(x)
\ei
\item the \emph{tensor product representation}
\bi{rrCl}
\rho_1\otimes \rho_2 \cl & L &\xrightarrow{\sim} &\End(V_1\times V_2)\\
& x & \mapsto & (\rho_1\otimes \rho_2) (x)  := \rho_1(x)\otimes \id_{V_2}+\id_{V_1}\otimes \rho_2(x).
\ei
\een
\ed

\be
The direct sum representation $\rho_{\mathrm{vec}}\oplus \rho_{\mathrm{spin}}\cl \so(3,\R)\xrightarrow{\sim}\End(\R^3\oplus\C^2)$ given in block-matrix form by
\bse
(\rho_{\mathrm{vec}}\oplus \rho_{\mathrm{spin}})(x) = \left(\begin{array}{c|c} \rho_{\mathrm{vec}}(x) & 0 \\ \hline 0 & \rho_{\mathrm{spin}}(x)\end{array}\right)
\ese
is a $7$-dimensional representation of $\so(3,\R)$.
\ee

\bd
A representation $\rho\cl L \xrightarrow{\sim} \End(V)$ is called \emph{reducible} if there exists a non-trivial vector subspace $U\se V$ which is invariant under the action of $\rho$, i.e.\
\bse
\forall \, x\in L: \forall \, u\in U : \ \rho(x)u\in U.
\ese
In other words, $\rho$ restricts to a representation $\rho|_U\cl L \xrightarrow{\sim} \End(U)$. 
\ed
\bd
A representation is \emph{irreducible} if it is not reducible.
\ed
\be
\ben[label=\roman*)]
\item The representation $\rho_{\mathrm{vec}}\oplus \rho_{\mathrm{spin}}\cl \so(3,\R)\xrightarrow{\sim}\End(\R^3\oplus\C^2)$ is reducible since, for example, we have a subspace $\R^3\oplus 0$ such that
\bse
\forall \, x \in \so(3,\R): \forall \, u\in \R^3\oplus 0 : \ (\rho_{\mathrm{vec}}\oplus \rho_{\mathrm{spin}}) (x)u\in\R^3\oplus 0.
\ese
\item The representations $\rho_{\mathrm{vec}}$ and $\rho_{\mathrm{spin}}$ are both irreducible.
\een
\ee

\br
Just like the simple Lie algebras are the building blocks of all semi-simple Lie algebras, the irreducible representations of a semi-simple Lie algebra are the building blocks of all finite-dimensional representations of the Lie algebra. Any such representation con be decomposed as the direct sum of irreducible representations, which can then be classified according to their so-called \emph{highest weights}.
\er

\subsection{The Casimir operator}

To every representation $\rho$ of a compact Lie algebra (i.e.\ the Lie algebra of a compact Lie group) there is associated an operator $\Omega_\rho$, called the Casimir operator. We will need some preparation in order to define it.

\bd
Let $\rho\cl L \xrightarrow{\sim} \End(V)$ be a representation of a complex Lie algebra $L$. We define the \emph{$\rho$-Killing form}\index{Killing form} on $L$ as 
\bi{rrCl}
\kappa_\rho \cl & L \times L & \xrightarrow{\sim} & \C\\
& (x,y) & \mapsto & \kappa_\rho(x,y) := \tr(\rho(x)\circ\rho(y)).
\ei
\ed
Of course, the Killing form we have considered so far is just $\kappa_{\ad}$. Similarly to $\kappa_{\ad}$, every $\kappa_\rho$ is symmetric and associative with respect to the Lie bracket of $L$. 
\bp
Let $\rho\cl L \xrightarrow{\sim} \End(V)$ be a faithful representation of a complex semi-simple Lie algebra $L$. Then, $\kappa_\rho$ is non-degenerate.
\ep
Hence, $\kappa_\rho$ induces an isomorphism $L\xrightarrow{\sim}L^*$ via
\bse
L \ni x \mapsto \kappa_\rho(x,-) \in L^*.
\ese
Recall that if $\{X_1,\ldots,X_{\dim L}\}$ is a basis of $L$, then the dual basis $\{\widetilde X^1,\ldots,\widetilde X^{\dim L}\}$ of $L^*$ is defined by
\bse
\widetilde X^i(X_j) = \delta_j^i.
\ese
By using the isomorphism induced by $\kappa_\rho$, we can find some $\xi_1,\ldots,\xi_{\dim L}\in L$ such that we have $\kappa(\xi_i,-)=\widetilde X^i$ or, equivalently,
\bse
\forall \, x\in L : \ \kappa_{\rho}(x,\xi_i) = \widetilde X^i(x).
\ese
We thus have
\bse
\kappa_\rho(X_i,\xi_j) = \delta_{ij} := \begin{cases}1 & \text{if }i\neq j\\ 0 & \text{otherwise.}\end{cases}
\ese

\bp
Let $\{X_i\}$ and $\{\xi_j\}$ be defined as above. Then
\bse
[X_j,\xi_k] = \sum_{m = 1}^{\dim L} C^{k}_{\phantom{k}mj}\xi_m, 
\ese
where $C^{k}_{\phantom{k}mj}$ are the structure constants with respect to $\{X_i\}$.
\ep

\bq
By using the associativity of $\kappa_\rho$, we have
\bse
\kappa_\rho(X_i,[X_j,\xi_k]) = \kappa_\rho([X_i,X_j],\xi_k) = C^{m}_{\phantom{m}ij} \kappa_\rho(X_m,\xi_k) = C^{m}_{\phantom{m}ij} \delta_{mk} = C^{k}_{\phantom{k}ij}.
\ese
But we also have
\bse
\kappa_\rho\Bigl(X_i,\sum_{m=1}^{\dim L}C^{k}_{\phantom{k}mj} \xi_m \Bigr) = \sum_{m=1}^{\dim L}C^{k}_{\phantom{k}mj} \kappa_\rho(X_i,\xi_m)  = \sum_{m=1}^{\dim L}C^{k}_{\phantom{k}mj} \delta_{im} = C^{k}_{\phantom{k}ij}.
\ese
Therefore
\bse
\forall \, 1\leq i \leq \dim L : \ \kappa_\rho\Bigl(X_i,[X_j,\xi_k]-\sum_{m=1}^{\dim L}C^{k}_{\phantom{k}mj} \xi_m \Bigr) = 0
\ese
and hence, the result follows from the non-degeneracy of $\kappa_{\rho}$.
\eq
We are now ready to define the Casimir operator and prove the subsequent theorem.
\bd
Let $\rho\cl L \xrightarrow{\sim} \End(V)$ be a faithful representation of a complex (compact) Lie algebra $L$  and let $\{X_1,\ldots,X_{\dim L}\}$ be a basis of $L$. The \emph{Casimir operator}\index{Casimir operator} associated to the representation $\rho$ is the endomorphism $\Omega_\rho\cl V \xrightarrow{\sim} V$
\bse
\Omega_\rho := \sum_{i=1}^{\dim L} \rho(X_i) \circ \rho(\xi_i).
\ese
\ed

\begin{theorem}
Let $\Omega_\rho$ the Casimir operator of a representation $\rho\cl L \xrightarrow{\sim} \End(V)$. Then
\bse
\forall \, x \in L : \ [\Omega_\rho,\rho(x)] = 0,
\ese
that is, $\Omega_\rho$ commutes with every endomorphism in $\im_\rho(L)$.
\end{theorem}

\bq
Note that the bracket above is that on $\End(V)$. Let $x=x^kX_k\in L$. Then
\bi{rCl}
[\Omega_\rho,\rho(x)] & = &  \biggl[\, \sum_{i=1}^{\dim L} \rho(X_i) \circ \rho(\xi_i), \rho(x^kX_k)\biggr]\\
& = & \sum_{i,k=1}^{\dim L} x^k [\rho(X_i) \circ \rho(\xi_i), \rho(X_k)].
\ei
Observe that if the Lie bracket as the commutator with respect to an associative product, as is the case for $\End(V)$, we have
\bi{rCl}
[AB,C] & = & ABC - CBA \\
& = & ABC - CBA -ACB +ACB \\
& = & A[B,C] + [A,C]B. 
\ei
Hence, by applying this, we obtain
\bi{rCl}
\sum_{i,k=1}^{\dim L} x^k [\rho(X_i) \circ \rho(\xi_i), \rho(X_k)] & = & \sum_{i,k=1}^{\dim L} x^k \bigl(\rho(X_i) \circ [\rho(\xi_i), \rho(X_k)]+[\rho(X_i) , \rho(X_k)]\circ \rho(\xi_i)\bigr)\\
 & = & \sum_{i,k=1}^{\dim L} x^k \bigl(\rho(X_i) \circ \rho([\xi_i, X_k])+\rho([X_i,X_k])\circ \rho(\xi_i)\bigr)\\
 & = & \sum_{i,k,m=1}^{\dim L} x^k \bigl(\rho(X_i) \circ \rho(-C^{i}_{\phantom{i}mk}\xi_m)+\rho(C^{m}_{\phantom{m}ik}X_m)\circ \rho(\xi_i)\bigr)\\
 & = & \sum_{i,k,m=1}^{\dim L} x^k \bigl(-C^{i}_{\phantom{i}mk}\rho(X_i) \circ \rho(\xi_m)+C^{m}_{\phantom{m}ik}\rho(X_m)\circ \rho(\xi_i)\bigr)\\
 & = & \sum_{i,k,m=1}^{\dim L} x^k \bigl(-C^{i}_{\phantom{i}mk}\rho(X_i) \circ \rho(\xi_m)+C^{i}_{\phantom{i}mk}\rho(X_i)\circ \rho(\xi_m)\bigr)\\
& = & 0,
\ei
where we have swapped the dummy summation indices $i$ and $m$ in the second term.
\eq

\begin{lemma}[Schur]
If $\rho\cl L \xrightarrow{\sim} \End(V)$ is irreducible, then any operator $S$ which commutes with every endomorphism in $\im_\rho(L)$ has the form
\bse
S = c_\rho \id_V
\ese
for some constant $c_\rho\in \C$ (or $\R$, if $L$ is a real Lie algebra).
\end{lemma}
It follows immediately that $\Omega_\rho = c_\rho \id_V$ for some $c_\rho$ but, in fact, we can say more.
\bp
The Casimir operator of $\rho\cl L \xrightarrow{\sim} \End(V)$ is $\Omega_\rho = c_\rho \id_V$, where
\bse
c_\rho = \frac{\dim L}{\dim V}.
\ese
\ep
\bq
We have
\bse
\tr(\Omega_\rho) = \tr(c_\rho\id_V) = c_\rho \dim V
\ese
and
\bi{rCl}
\tr(\Omega_\rho) & = & \tr \biggl(\, \sum_{i=1}^{\dim L} \rho(X_i) \circ \rho(\xi_i) \biggr)\\
 & = &  \sum_{i=1}^{\dim L} \tr(\rho(X_i) \circ \rho(\xi_i) )\\
 & = &  \sum_{i=1}^{\dim L} \kappa_\rho(X_i,\xi_i)\\
 & = &  \sum_{i=1}^{\dim L} \delta_{ii}\\
 & = &  \dim L ,
\ei
which is what we wanted.
\eq

\be
Consider the Lie algebra $\so(3,\R)$ with basis $\{J_1,J_2,J_3\}$ satisfying
\bse
[J_i,J_j] = \varepsilon_{ijk}J_k,
\ese
where we assume the summation convention on the lower index $k$. Recall that the representation $\rho_{\mathrm{vec}}\cl\so(3,\R)\xrightarrow{\sim}\End(\R^3)$ is defined by
\bse
\rho_{\mathrm{vec}}(J_1) := \begin{pmatrix}0 & 0 & 0\\ 0 & 0 & -1\\ 0 & 1 & 0\end{pmatrix}, \qquad \rho_{\mathrm{vec}}(J_2) := \begin{pmatrix}0 & 0 & 1\\ 0 & 0 & 0\\ -1 & 0 & 0\end{pmatrix}, \qquad \rho_{\mathrm{vec}}(J_3) :=\begin{pmatrix}0 & -1 & 0\\ 1 & 0 & 0\\ 0 & 0 & 0\end{pmatrix}.
\ese
Let us first evaluate the components of $\kappa_{\rho_{\mathrm{vec}}}$. We have
\bi{rCl}
(\kappa_{\rho_{\mathrm{vec}}})_{11} := \kappa_{\rho_{\mathrm{vec}}}(J_1,J_1) & = & \tr(\rho_{\mathrm{vec}}(J_1)\circ \rho_{\mathrm{vec}}(J_1)) \\
& = & \tr((\rho_{\mathrm{vec}}(J_1))^2)\\
& = & \tr\begin{pmatrix}0 & 0 & 0\\ 0 & 0 & -1\\ 0 & 1 & 0\end{pmatrix}^{\negmedspace 2}\\
& = & \tr\begin{pmatrix}0 & 0 & 0\\ 0 & -1 & 0\\ 0 & 0 & -1\end{pmatrix}\\
& = & -2.
\ei
After calculating the other components similarly, we find
\bse
[(\kappa_{\rho_{\mathrm{vec}}})_{ij}] = \begin{pmatrix}-2 & 0 & 0\\ 0 & -2 & 0\\ 0 & 0 & -2\end{pmatrix}.
\ese
Thus, $\kappa_{\rho_{\mathrm{vec}}}(J_i,\xi_j)=\delta_{ij}$ requires that we define $\xi_i := -\tfrac{1}{2} J_i$. Then, we have
\bi{rCl}
\Omega_{\rho_{\mathrm{vec}}} & := & \sum_{i=1}^{3} \rho_{\mathrm{vec}}(J_i) \circ \rho_{\mathrm{vec}}(\xi_i)\\
& = & \sum_{i=1}^{3} \rho_{\mathrm{vec}}(J_i) \circ \rho_{\mathrm{vec}}(-\tfrac{1}{2} J_i)\\
& = & -\frac{1}{2} \sum_{i=1}^{3} (\rho_{\mathrm{vec}}(J_i))^2\\
& = & -\frac{1}{2} \left( \begin{pmatrix}0 & 0 & 0\\ 0 & 0 & -1\\ 0 & 1 & 0\end{pmatrix}^{\negmedspace 2}+ \begin{pmatrix}0 & 0 & 1\\ 0 & 0 & 0\\ -1 & 0 & 0\end{pmatrix}^{\negmedspace 2} +\begin{pmatrix}0 & -1 & 0\\ 1 & 0 & 0\\ 0 & 0 & 0\end{pmatrix}^{\negmedspace 2}\ \right)\\
& = & -\frac{1}{2} \left( \begin{pmatrix}0 & 0 & 0\\ 0 & -1 & 0\\ 0 & 0 & -1\end{pmatrix}+ \begin{pmatrix}-1 & 0 & 0\\ 0 & 0 & 0\\ 0 & 0 & -1\end{pmatrix} +\begin{pmatrix}-1 & 0 & 0\\ 0 & -1 & 0\\ 0 & 0 & 0\end{pmatrix} \right)\\
& = & \begin{pmatrix}1 & 0 & 0\\ 0 & 1 & 0\\ 0 & 0 & 1\end{pmatrix}.
\ei
Hence $\Omega_{\rho_{\mathrm{vec}}}=c_{\rho_{\mathrm{vec}}}\id_{\R^3}$ with $c_{\rho_{\mathrm{vec}}} = 1$, which agrees with our previous theorem since
\bse
\frac{\dim \so(3,\R)}{\dim \R^3} = \frac{3}{3} = 1.
\ese
\ee

\be
Let us consider the Lie algebra $\so(3,\R)$ again, but this time with representation $\rho_{\mathrm{spin}}$. Recall that this is given by
\bse
\rho_{\mathrm{spin}}(J_1) := -\frac{\mathrm{i}}{2}\, \sigma_1, \qquad \rho_{\mathrm{spin}}(J_2) := -\frac{\mathrm{i}}{2}\, \sigma_2, \qquad \rho_{\mathrm{spin}}(J_3) := -\frac{\mathrm{i}}{2}\, \sigma_3,
\ese
where $\sigma_1,\sigma_2,\sigma_3$ are the Pauli matrices. Recalling that $\sigma_1^2=\sigma_2^2=\sigma_3^2=\id_{\C^2}$, we calculate
\bi{rCl}
(\kappa_{\rho_{\mathrm{spin}}})_{11} := \kappa_{\rho_{\mathrm{spin}}}(J_1,J_1) & = & \tr(\rho_{\mathrm{spin}}(J_1)\circ \rho_{\mathrm{spin}}(J_1)) \\
& = & \tr((\rho_{\mathrm{spin}}(J_1))^2)\\
& = &(-\tfrac{\mathrm{i}}{2})^2 \tr(\sigma_1^{2})\\
& = & -\tfrac{1}{4} \tr(\id_{\C^2})\\
& = & -1.
\ei
Note that $\tr(\id_{\C^2})=4$, since $\tr(\id_V)=\dim V$ and here $\C^2$ is considered as a $4$-dimensional vector space over $\R$. Proceeding similarly, we find that the components of $\kappa_{\rho_{\mathrm{spin}}}$ are
\bse
[(\kappa_{\rho_{\mathrm{spin}}})_{ij}] = \begin{pmatrix}-1 & 0 & 0\\ 0 & -1 & 0\\ 0 & 0 & -1\end{pmatrix}.
\ese
Hence, we define $\xi_i := - J_i$. Then, we have
\bi{rCl}
\Omega_{\rho_{\mathrm{spin}}} & := & \sum_{i=1}^{3} \rho_{\mathrm{spin}}(J_i) \circ \rho_{\mathrm{spin}}(\xi_i)\\
& = & \sum_{i=1}^{3} \rho_{\mathrm{spin}}(J_i) \circ \rho_{\mathrm{spin}}(-J_i)\\
& = & - \sum_{i=1}^{3} (\rho_{\mathrm{spin}}(J_i))^2\\
& = & - \Bigl(-\frac{\mathrm{i}}{2}\Bigr)^2\, \sum_{i=1}^{3} \sigma_i^2\\
& = & \frac{1}{4} \sum_{i=1}^{3} \id_{\C^2}\\
& = & \frac{3}{4}\id_{\C^2},
\ei
in accordance with the fact that
\bse
\frac{\dim \so(3,\R)}{\dim \C^2} = \frac{3}{4}.
\ese
\ee

\subsection{Representations of Lie groups}

We now turn to representations of Lie groups. Given a vector space $V$, recall that the subset of $\End(V)$ consisting of the invertible endomorphisms and denoted
\bse
\GL(V) \equiv \Aut(V):= \{\phi\in \End(V)\mid \det \phi \neq 0\},
\ese
forms a group under composition, called the automorphism group (or general linear group) of $V$. Moreover, if $V$ is a finite-dimensional $K$-vector space, then $V\cong_{\mathrm{vec}}K^{\dim V}$ and hence the group $\GL(V)$ can be given the structure of a Lie group via
\bse
\GL(V)\cong_{\mathrm{Lie \, grp}}\GL(K^{\dim V}):=\GL(\dim V,K).
\ese
This is, of course, if we have established a topology and a differentiable structure on $K^d$, as is the case for $\R^d$ and $\C^d$.
\bd
A \emph{representation}\index{representation} of a Lie group $(G,\bullet)$ is a Lie group homomorphism
\bse
R\cl G \to \GL(V)
\ese
for some finite-dimensional vector space $V$.
\ed
Recall that $R\cl G \to \GL(V)$ is a Lie group homomorphism if it is smooth and
\bse
\forall \, g_1,g_2\in G : R(g_1\bullet g_2) = R(g_1)\circ R( g_2).
\ese
Note that, as is the case with any group homomorphism, we have
\bse
R(e) = \id_V \qquad \text{and}\qquad R(g^{-1}) = R(g)^{-1}.
\ese
\be
Consider the Lie group $\SO(2,\R)$. As a smooth manifold, $\SO(2,\R)$ is isomorphic to the circle $S^1$. Let $U=S^1\sm\{p_0\}$, where $p$ is any point of $S^1$, so that we can define a chart $\theta\cl U \to [0,2\pi)\se\R$ on $S^1$ by mapping each point in $U$ to an ``angle'' in $[0,2\pi)$.
\bse
\begin{tikzpicture}[scale=0.9]
\draw[thick] (0,0) circle [radius=2];
\draw (0,0) -- (2,0);
\draw (0,0) -- (2*cos 60, 2* sin 60) node[above right=-1pt] {$p$};
\filldraw[fill=lightergray] (0,0) -- (0.6,0) arc (0:60:0.6) -- cycle;
\draw (1,0.5) node {$\theta(p)$};
\draw (-1.8,1.8) node {$U$};
\draw[thick,fill=white] (2,0) circle [radius=0.11] node[right=3pt] {$p_0$};
\end{tikzpicture}
\ese
The operation
\bse
p_1\bullet p_2:=(\theta(p_1)+\theta(p_2))\! \mod 2\pi
\ese
endows $S^1\cong_{\mathrm{diff}}\SO(2,\R)$ with the structure of a Lie group. Then, a representation of $\SO(2,\R)$ is given by
\bi{rrCl}
R\cl & \SO(2,\R) & \to & \GL(\R^2)\\
&  p & \mapsto & \biggl( \begin{matrix} \cos \theta(p) & \sin \theta(p) \\ -\sin \theta(p) & \cos \theta(p) \end{matrix}\biggr).
\ei
Indeed, the addition formul\ae for sine and cosine imply that
\bse
R(p_1\bullet p_2) = R(p_1)\circ R(p_2).   
\ese
\ee

\be
Let $G$ be a Lie group (we suppress the $\bullet$ in this example). For each $g\in G$, define the Adjoint map
\bi{rrCl}
\Ad_g \cl & G & \to & G\\
& h & \mapsto & g h g^{-1}.
\ei
Note the capital ``A'' to distinguish this from the adjoint map on Lie algebras. Since $\Ad_g$ is a composition of the Lie group multiplication and inverse map, it is a smooth map. Moreover, we have
\bse
\Ad_g(e) = geg^{-1} = gg^{-1} = e.
\ese
Hence, the push-forward of $\Ad_g$ at the identity is the map
\bse
({\Ad_g}_*)_e \cl  T_eG  \xrightarrow{\sim}   T_{\Ad_g(e)}G=T_eG.
\ese
Thus, we have $\Ad_g\in\End(T_eG)$. In fact, you can check that
\bse
({\Ad_{g^{-1}}}_*)_e\circ ({\Ad_g}_*)_e = ({\Ad_g}_*)_e\circ ({\Ad_{g^{-1}}}_*)_e = \id_{T_eG},
\ese
and hence we have, in particular, $\Ad_g\in\GL(T_eG)\cong_{\mathrm{Lie\,grp}}\GL(\mathcal{L}(G))$.

\noindent We can therefore construct a map
\bi{rrCl}
\Ad \cl & G & \to & \GL(T_eG)\\
& g & \mapsto & {\Ad_g}_*
\ei
which, as you can check, is a representation of $G$ on its Lie algebra.
\ee

\br
Since a representation $R$ of a Lie group $G$ is required to be smooth, we can always consider its differential or push-forward at the identity
\bse
(R_*)_e \cl T_eG \xrightarrow{\sim} T_{\id_V}\!\GL(V).
\ese
Since for any $A,B\in T_eG$ we have
\bse
(R_*)_e[A,B] =[(R_*)_eA,(R_*)_eB],
\ese
the map $(R_*)_e$ is a representation of the Lie algebra of $G$ on the vector space $\GL(V)$. In fact, in the previous example we have 
\bse
(\Ad_*)_e = \ad,
\ese
where $\ad$ is the adjoint representation of $T_eG$.
\er





















