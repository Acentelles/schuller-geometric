

We have seen in detail how to construct a Lie algebra from a given Lie group. We would now like to consider the inverse question, i.e.\ whether, given a Lie algebra, we can construct a Lie group whose associated Lie algebra is the given one and, if this is the case, whether this correspondence is bijective.

% We will find that the answer to the first question is affirmative. Given a Lie algebra, we will construct a Lie group via something called the exponential map. However, we can already answer in the negative the second question. The Lie groups
% \bse
% \Ort(n,\R) := \{\phi\in \GL(n,\R)\}
% \ese
% and $\SU(2,\C)$ are an example of Lie groups which are not isomorphic but give rise to the same Lie algebra. Hence, the correspondence between Lie groups and Lie algebras cannot be bijective. 

\subsection{Integral curves}

\bd
Let $M$ be a smooth manifold and let $Y\in \Gamma(TM)$. An \emph{integral curve}\index{integral curve} of $Y$ is smooth curve $\gamma\cl(-\varepsilon,\varepsilon)\to M$, with $\varepsilon > 0$, such that
\bse
\forall \, \lambda \in (-\varepsilon,\varepsilon) : \ X_{\gamma,\gamma(\lambda)} = Y|_{\gamma(\lambda)}.
\ese
\ed

It follows from the local existence and uniqueness of solutions to ordinary differential equations that, given any $Y\in \Gamma(TM)$ and any $p\in M$, there exist $\varepsilon >0$ and a smooth curve $\gamma\cl (-\varepsilon,\varepsilon) \to M$ with $\gamma(0)=p$ which is an integral curve of $Y$. 

Moreover, integral curves are locally unique. By this we mean that if $\gamma_1$ and $\gamma_2$ are both integral curves of $Y$ through $p$, i.e.\ $\gamma_1(0)=\gamma_2(0) = p$, then $\gamma_1=\gamma_2$ on the intersection of their domains of definition. We can get genuine uniqueness as follows.

\bd
The \emph{maximal integral curve} of $Y\in \Gamma(TM)$ through $p\in M$ is the unique integral curve $\gamma\cl I^p_{\mathrm{max}}\to M$ of $Y$ through $p$, where
\bse
I_{\mathrm{max}}^{p} := \bigcup\{ I \se \R \mid \text{there exists an integral curve }\gamma\cl I \to M \text{ of $Y$ through $p$}\}.
\ese
\ed
For a given vector field, in general, $I_{\mathrm{max}}^{p}$ will differ from point to point. 
\bd
A vector field is \emph{complete} if $I_{\mathrm{max}}^{p}=\R$ for all $p\in M$.
\ed
We have the following result.
\begin{theorem}
On a compact manifold, every vector field is complete.
\end{theorem}
On a Lie group, even if non-compact, there are always complete vector fields.
\begin{theorem}
Every left-invariant vector field on a Lie group is complete.
\end{theorem}
The maximal integral curves of left-invariant vector fields are crucial in the construction of the map that allows us to go from a Lie algebra to a Lie group.

\subsection{The exponential map}

Let $G$ be a Lie group. Recall that given any $A\in T_eG$, we can define the uniquely determined left-invariant vector field $X^A:=j(A)$ via the isomorphism $j\cl T_eG \xrightarrow{\sim}\mathcal{L}(G)$ as
\bse
X^A|_g := (\ell_g)_* (A).
\ese
Then let $\gamma^A\cl \R \to G$ be the maximal integral curve of $X^A$ through $e\in G$.
\bd
Let $G$ be a Lie group. The \emph{exponential map}\index{exponential map} is defined as
\bi{rrCl}
\exp \cl & T_eG & \to & G\\
& A & \mapsto & \exp(A):=\gamma^A(1)
\ei
\ed
\begin{theorem}
\ben[label=\roman*)]
\item The map $\exp$ is smooth and a local diffeomorphism around $0\in T_eG$, i.e.\ there exists an open $V\se T_eG$ containing $0$ such that the restriction
\bse
\exp|_V\cl V \to \exp(V) \se G
\ese
is bijective and both $\exp|_V$ and $(\exp|_V)^{-1}$ are smooth.
\item If $G$ is compact, then $\exp$ is surjective.
\een
\end{theorem}
Note that the maximal integral curve of $X^0$ is the constant curve $\gamma^0(\lambda)\equiv e$, and hence we have $\exp(0)=e$. Then first part of the theorem then says that we can recover a neighbourhood of the identity of $G$ from a neighbourhood of the identity of $T_eG$.

Since $T_eG$ is a vector space, it is non-compact (intuitively, it extends infinitely far away in every direction) and hence, if $G$ is compact, $\exp$ cannot be injective. This is because, by the second part of the theorem, it would then be a diffeomorphism $T_eG\to G$. But as $G$ is compact and $T_eG$ is not, they are not diffeomorphic.

\bp
Let $G$ be a Lie group. The image of $\exp\cl T_eG\to G$ is the connected component of $G$ containing the identity.
\ep
Therefore, if $G$ itself is connected, then $\exp$ is again surjective. Note that, in general, there is no relation between connected and compact topological spaces, i.e.\ a topological space can be either, both, or neither.

\be
Let $B\cl V\times V$ be a pseudo inner product on $V$. Then
\bse
\Ort(V) := \{\phi\in \GL(V)\mid \forall \, v,w\in V : B(\phi(v),\phi(w))=B(v,w)\}
\ese
is called the \emph{orthogonal group} of $V$ with respect to $B$. Of course, if $B$ or the base field of $V$ need to be emphasised, they can be included in the notation. Every $\phi\in \Ort(V)$ has determinant $1$ or $-1$. Since $\det$ is multiplicative, we have a subgroup
\bse
\SO(V) := \{\phi\in \Ort(V)\mid \det\phi = 1\}.
\ese
These are, in fact, Lie subgroups of $\GL(V)$. The Lie group $\SO(V)$ is connected while
\bse
\Ort(V)=\SO(V)\cup \{\phi\in \Ort(V)\mid \det \phi = -1\}
\ese
is disconnected. Since $\SO(V)$ contains $\id_V$, we have
\bse
\so(V) := T_{\id_V}\!\SO(V) = T_{\id_V}\!\Ort(V) =: \ort(V)
\ese
and
\bse
\exp(\so(V))=\exp(\ort(V))=\SO(V).
\ese
\ee

\be
Choosing a basis $A_1,\ldots,A_{\dim G}$ of $T_eG$ often provides a convenient co-ordinatisation of $G$ near $e$. Consider, for example, the Lorentz group
\bse
\Ort(3,1) \equiv \Ort(\R^4) = \{\Lambda\in \GL(\R^4)\mid \forall \, x,y\in \R^4 :  B(\Lambda(x),\Lambda(y))=B(x,y) \},
\ese
where $B(x,y):=\eta_{\mu\nu}x^\mu y^\nu$, with $0\leq \mu,\nu\leq 3$ and
\bse
[\eta^{\mu\nu}] = [\eta_{\mu\nu}] := \left( 
  \begin{matrix}
   -1 & 0 & 0 & 0 \\
    0 & 1 & 0 & 0 \\
    0 & 0 & 1 & 0 \\
    0 & 0 & 0 & 1
  \end{matrix}\right).
\ese
The Lorentz group $\Ort(3,1)$ is $6$-dimensional, hence so is the Lorentz algebra $\ort(3,1)$. For convenience, instead of denoting a basis of $\ort(3,1)$ as $\{M^i\mid i=1,\ldots,6\}$, we will denote it as $\{M^{\mu\nu}\mid 0\leq \mu,\nu\leq 3\}$ and require that the indices $\mu,\nu$ be anti-symmetric, i.e.\
\bse
M^{\mu\nu} = -M^{\nu\mu }.
\ese
Then $M^{\mu\nu}=0$ when $\rho=\sigma$, and the set $\{M^{\mu\nu}\mid 0\leq \mu,\nu\leq 3\}$, while technically not linearly independent, contains the 6 independent elements that we want to consider as a basis. These basis elements satisfy the following bracket relation
\bse
[M^{\mu\nu},M^{\rho\sigma}] =\eta^{\nu\sigma} M^{\mu\rho} +\eta^{\mu\rho} M^{\nu\sigma} -\eta^{\nu\rho} M^{\mu\sigma} -\eta^{\mu\sigma} M^{\nu\rho} . 
\ese
Any element $\lambda\in \ort(3,1)$ can be expressed as linear combination of the $M^{\mu\nu}$,
\bse
\lambda = \tfrac{1}{2}\omega_{\mu\nu}M^{\mu\nu}
\ese
where the indices on the coefficients $\omega_{\mu\nu}$ are also anti-symmetric, and the factor of $\tfrac{1}{2}$ ensures that the sum over all $\mu,\nu$ counts each anti-symmetric pair only once. Then, we have
\bse
\Lambda = \exp(\lambda) = \exp(\tfrac{1}{2}\omega_{\mu\nu}M^{\mu\nu}) \in \Ort(3,1).
\ese
The subgroup of $\Ort(3,1)$ consisting of the the space-orientation preserving Lorentz transformations, or \emph{proper} Lorentz transformations, is denoted by $\SO(3,1)$. The subgroup consisting of the time-orientation preserving, or \emph{orthochronous}, Lorentz transformations is denoted by $\Ort^+(3,1)$. The Lie group $\Ort(3,1)$ is disconnected: its four connected components are
\ben[label=\roman*)]
\item $\SO^+(3,1):=\SO(3,1)\cap\Ort^+(3,1)$, also called the \emph{restricted Lorentz group}, consisting of the proper orthochronous Lorentz transformations;
\item $\SO(3,1)\sm \Ort^+(3,1)$, the proper non-orthochronous transformations;
\item $\Ort^+(3,1)\sm\SO(3,1)$, the improper orthochronous transformations;
\item $\Ort(3,1)\sm (\SO(3,1)\cup\Ort^+(3,1))$, the improper non-orthochronous transformations.
\een
Since $\id_{\R^4}\in \SO^+(3,1)$, we have $\exp(\ort(3,1))=\SO^+(3,1)$. Then $\{M^{\mu\nu}\}$ provides a nice co-ordinatisation of $\SO^+(3,1)$ since, if we choose
\bse
[\omega_{\mu\nu}] = \left(
  \begin{matrix}
        0   &   \psi_1   &   \psi_2   &   \psi_3   \\
    -\psi_1 &      0     &  \varphi_3 & -\varphi_2 \\
    -\psi_2 & -\varphi_3 &      0     &  \varphi_1 \\
    -\psi_3 &  \varphi_2 & -\varphi_1 &      0
  \end{matrix}
\right)
\ese
then the Lorentz transformation $\exp(\tfrac{1}{2}\omega_{\mu\nu}M^{\mu\nu})\in \SO^+(3,1)$ corresponds to a boost in the $(\psi_1,\psi_2,\psi_3)$ direction and a space rotation by $(\varphi_1,\varphi_2,\varphi_3)$. Indeed, in physics one often thinks of the Lie group $\SO^+(3,1)$ as being generated by $\{M^{\mu\nu}\}$.

A representation $\rho\cl T_{\id_{\R^4}}\!\SO^+(3,1)\xrightarrow{\sim} \End(\R^4)$ is given by
\bse
\rho(M^{\mu\nu})^a_{\phantom{a}b} := \eta^{\nu a}\delta^{\mu}_b - \eta^{\mu a}\delta^{\nu}_b 
\ese
which is probably how you have seen the $M^{\mu\nu}$ themselves defined in some previous course on relativity theory. Using this representation, we get a corresponding representation
\bi{rrCl}
R\cl & \SO^+(3,1) \to \GL(\R^4)
\ei
via the exponential map by defining
\bse
R(\Lambda) = \exp(\tfrac{1}{2}\omega_{\mu\nu}\rho(M^{\mu\nu})).
\ese
Then, the map $\exp$ becomes the usual exponential (series) of matrices.
\ee




\bd
A \emph{one-parameter subgroup}\index{one-parameter subgroup} of a Lie group $G$ is a Lie group homomorphism
\bse
\xi \cl \R \to G,
\ese
with $\R$ understood as a Lie group under ordinary addition.
\ed

\be
Let $M$ be a smooth manifold and let $Y\in\Gamma(TM)$ be a complete vector field. The \emph{flow} of $Y$ is the smooth map
\bi{rrCl}
\Theta \cl & \R\times M & \to & M\\
& (\lambda,p) & \mapsto & \Theta_\lambda(p):= \gamma_p(\lambda),
\ei
where $\lambda_p$ is the maximal integral curve of $Y$ through $p$. For a fixed $p$, we have
\bse
\Theta_{0} = \id_M, \qquad \Theta_{\lambda_1}\circ \Theta_{\lambda_2} = \Theta_{\lambda_1+\lambda_2}, \qquad \Theta_{-\lambda} = \Theta_{\lambda}^{-1}.
\ese
For each $\lambda\in \R$, the map $\Theta_\lambda$ is a diffeomorphism $M\to M$. Denoting by $\mathrm{Diff}(M)$ the group (under composition) of the diffeomorphisms $M\to M$, we have that the map 
\bi{rrCl}
\xi \cl & \R & \to & \mathrm{Diff}(M)\\
& \lambda & \mapsto & \Theta_\lambda
\ei
is a one-parameter subgroup of $\mathrm{Diff}(M)$.
\ee

\begin{theorem}
Let $G$ be a Lie group.
\ben[label=\roman*)]
\item Let $A\in T_eG$. The map
\bi{rrCl}
\xi^A\cl & \R & \to & G\\
& \lambda & \mapsto & \xi^A(\lambda):=\exp(\lambda A)
\ei
is a one-parameter subgroup.
\item Every one-parameter subgroup of $G$ has the form $\xi^A$ for some $A\in T_eG$.
\een
\end{theorem}

Therefore, the Lie algebra allows us to study all the one-parameter subgroups of the Lie group. 

\begin{theorem}
Let $G$ and $H$ be Lie groups and let $\phi\cl G \to H$ be a Lie group homomorphism. Then, for all $A\in T_{e_G}G$, we have
\bse
\phi (\exp (A))= \exp ((\phi_*)_{{\scriptstyle e}_G}A).
\ese
Equivalently, the following diagram commutes.
\bse
\begin{tikzcd}
T_{e_G}G \ar[dd,"\exp"'] \ar[rr,"(\phi_*)_{{\scriptstyle e}_G}"]&& T_{e_H}H\ar[dd,"\exp"]\\
&&\\
G\ar[rr,"\phi"] && H
\end{tikzcd}
\ese
\end{theorem}
In particular, for $\phi\equiv \Ad_g\cl G\to G$, we have
\bse
\Ad_g (\exp(A)) = \exp(({\Ad_g}_*)_eA).
\ese










