Let $M$ be a set. We denote by $\cP(M)$ the set of subsets of $M$, called the \emph{power set} of $M$:
\bse
\cP(M) := \{S : S \se M\} .
\ese
\be
Let $M = \{a,b,c\}$. Then we have:
\bse
\cP(M)=\{\vn,\{a\},\{b\},\{c\},\{a,b\},\{a,c\},\{b,c\},\{a,b,c\}\}.
\ese
\ee
\bd
A \emph{topology} on $M$ is a set $\cO_M \se \cP(M)$ such that:
\ben
\item $\vn \in \cO_M$ and $M \in \cO_M$;
\item $\{S_i:1\leq i\leq n \} \se \cO_M \imp \bigcap_{i=1}^n S_i \in \cO_M$;
\item $\{S_\a:\a\in A\} \se \cO_M \imp \bigcup_{\a\in A} S_\a \in \cO_M$.
\een
\ed
In words, a topology $\cO_M$ on $M$ is a set of subsets of $M$ which contains $\vn$ and $M$ and is closed under the taking of \emph{finite} intersections and of \emph{arbitrary} unions.
\be
Let $M = \{a,b,c\}$. Then $\cO_M=\{\vn,\{a\},\{a,b\},\{a,b,c\}\}$ is a topology on $M$ while $\cO_M'=\{\vn,\{a\},\{b\},\{a,b,c\}\}$ is not since $\{a\}\cup\{b\}\notin\cO_M'$.
\ee
\be
Let $M$ be any set. Then $\cO_c=\{\vn,M\}$ and $\cO_d=\cP(M)$ are topologies on~$M$, called the \emph{chaotic} (or \emph{trivial}) topology and the \emph{discrete} topology, respectively.
\ee
\bd
Let $\cO_1$ and $\cO_2$ be two topologies on a set $M$. If $\cO_1 \ss \cO_2$, then we say that $\cO_1$ is a \emph{coarser} (or \emph{weaker}) topology than $\cO_2$. Equivalently, we say that $\cO_2$ is a \emph{finer} (or \emph{stronger}) topology than $\cO_1$.
\ed
Clearly, the chaotic topology is the coarsest topology on any given set, while the discrete topology is the finest.
\bd
A \emph{topological space} is a pair $(M,\cO_M)$, where $M$ is a set and $\cO_M$ a topology on $M$.
\ed
A topology on a set allows to define the concepts of open and closed sets.
\bd
Let $(M,\cO_M)$ be a topological space. A subset $S$ of $M$ is said to be \emph{open} if $S \in \cO_M$ and \emph{closed} if $M\sm S \in \cO_M$. 
\ed
Notice that the notions of open and closed sets, as defined, are not mutually exclusive. A set could be both or neither, or one and not the other.
\be
Let $(M,\cO_M)$ be a topological space. Then $\vn$ is open since $\vn \in \cO_M$. However, $\vn$ is also closed since $M\sm \vn = M \in \cO_M$. Similarly for $M$.
\ee
\be
Let $M = \{a,b,c\}$ and let $\cO_M=\{\vn,\{a\},\{a,b\},\{a,b,c\}\}$. Then $\{a\}$ is open but not closed, $\{b,c\}$ is closed but not open, and $\{b\}$ is neither open nor closed.
\ee
Topological spaces provide the minimal structure required to define continuity of maps.
\bd
Let $(M,\cO_M)$ and $(N,\cO_N)$ be topological spaces and let $f\cl M\to N$ be a map. We say that $f$ is \emph{continuous} (with respect to the topologies $\cO_M$ and $\cO_N$) if:
\bse
\forall \, S \in \cO_N \, , \ f^{-1}(S) \in \cO_M ,
\ese
where $f^{-1}(S) := \{m \in M : f(m) \in S\}$ is the \emph{pre-image} of $S$ under the map $f$.
\ed
Informally, one says that $f$ is continuous if the pre-images of open sets are open. 
\be
If $M$ is equipped with the discrete topology, or $N$ with the chaotic topology, then any map $f\cl M \to N$ is continuous. Indeed, let $S \in \cO_N$. If $\cO_M=\cP(M)$ (and $\cO_N$ is any topology), then we have:
\bse
f^{-1}(S) = \{m \in M : f(m) \in S\} \se M \in \cP(M) = \cO_M.
\ese
If instead $\cO_N=\{\vn,N\}$ (and $\cO_M$ is any topology), then either $S=\vn$ or $S=N$ and thus, we have:
\bse
f^{-1}(\vn) = \vn \in \cO_M \quad \t{and} \quad f^{-1}(N) = M \in \cO_M.
\ese
\ee
\be
Let $M = \{a,b,c\}$ and $N=\{1,2,3\}$, with respective topologies:
\bse
\cO_M=\{\vn,\{b\},\{a,c\},\{a,b,c\}\} \quad \t{and} \quad \cO_N=\{\vn,\{2\},\{3\},\{1,3\},\{2,3\},\{1,2,3\}\},
\ese
and let $f\cl M \to N$ by defined by:
\bse
f(a) = 2, \quad f(b)=1, \quad f(c)=2.
\ese
Then $f$ is continuous. Indeed, we have:
\begin{align*}
f^{-1}(\vn) &= \vn, & f^{-1}(\{2\}) &= \{a,c\}, & f^{-1}(\{3\}) &= \vn,\\ 
f^{-1}(\{1,3\}) &= \{b\}, & f^{-1}(\{2,3\}) &= \{a,c\}, & f^{-1}(\{1,2,3\}) &= \{a,b,c\},
\end{align*}
and hence $f^{-1}(S) \in \cO_M$ for all $S \in \cO_N$.
\ee






















